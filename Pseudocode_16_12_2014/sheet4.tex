\documentclass[11pt]{article}
\usepackage[T1]{fontenc}
\usepackage[utf8]{inputenc}
\usepackage[english]{babel}
\usepackage{amssymb}
\usepackage{amsmath}
\usepackage{amsthm}
\usepackage{graphicx}
\usepackage{hyperref}
\usepackage{lscape}
\usepackage{enumerate}
\usepackage{a4wide}
\usepackage{paralist}
\usepackage{url}
\usepackage{nopageno}
\usepackage{bbm}

%----------------------------
%Margins
%----------------------------
\topmargin -1.0 cm
\textheight 22 cm
\textwidth 17.0cm
\oddsidemargin -.5cm
\evensidemargin -.5cm
%----------------------------

\title{Discrete and Algorithmic Geometry \\Sheet 4}
\author{Clara Mateo Campo, Aitor Pérez Pérez, Arnau Planas Bah\'i}
\date{}

\newcommand{\cA}{\mathcal{A}}
\newcommand{\cS}{\mathcal{S}}
\newcommand{\R}{\mathbbm{R}}
\newcommand{\Z}{\mathbbm{Z}}
\newcommand{\VV}{\mathcal{V}}
\DeclareMathOperator{\conv}{conv}
\DeclareMathOperator{\cols}{cols}
\DeclareMathOperator{\Sl}{Sl}
\DeclareMathOperator{\igc}{igc}
\DeclareMathOperator{\Gale}{Gale}

\newcommand{\ojo}[1]{\textbf{\sffamily\boldmath{[#1]}}}


\newtheorem{theorem}{Teorema}[section]
\newtheorem{definition}[theorem]{Definition}
\newtheorem{lemma}[theorem]{Lema}
\newtheorem{proposition}[theorem]{Proposici\'o}
\newtheorem{corollary}[theorem]{Coro\lgem{ari}}
\newtheorem{example}[theorem]{Exemple}
\newtheorem{exercise}[theorem]{Exercise}
\newtheorem{solution}[theorem]{Soluci\'o}

\newtheorem*{problemG}{Problem G}
\newtheorem*{problemG*}{Problem G$^\star$}



\begin{document}
\maketitle

\begin{problemG*}
Enumerate, up to combinatorial equivalence, all balanced   configurations~$\VV$ of $n$~vectors in~$\Z^e$ whose coordinates are   all at most~$m$ in absolute value, such that
\begin{enumerate}[\qquad\upshape(1)]
\item the maximum $m$ is achieved by some $v\in\VV$, 
\item and such that no hyperplane spanned by $e-1$~of the vectors   strictly separates exactly one vector from the others.
\end{enumerate}
\end{problemG*}

\emph{For this, recall that a vector configuration $\VV = (v_1,\dots,v_n)$ is balanced if $\sum_i v_i=0$;
that no hyperplane defined by $e-1$~elements of~$\VV$ separates   exactly one vector from the others iff the Gale dual of $\VV$ is in   convex position; and that two vector configurations are combinatorially equivalent if they define the same oriented matroid.}
\\\\This problem can be divided in two parts
\begin{enumerate}
	\item Find all ``diferent'' vector configurations
	\item Identify those configurations that correspond to the same polytope.
\end{enumerate}
\textbf{Pseudocode}
\\Trivial algorithm: check all the possibilities and after that check if they are combinatorially equivalent. $\mathcal{O}(m^{e(n-1)})$. This is really inefficient!
\\Note that up to combinatorial equivalence we can reduce the number of possibilities to $\mathcal{O}(m^{e(n-1)}/(|BC_e|n!))$ and $|BC_e| = 2^e e!$, so, it can be done much more efficiently than the algorithm above.
\begin{enumerate}
\item Dynamic programming? Calculate the $\mathcal{V}(n,e,m)$ using all the other configurations $\mathcal{V}(n',e',m')$ where $n'<n$, $e'<e$ and $m'<m$. 

Basic cases:
For $m=0$, the only configuration we can choose is $n$ zero vectors.
For $n=1$, we can take every possible vector.
(Estic molt esp\`es i no se m'acudeixen altres casos base, a banda $e=0$, que \'es una parida i no semblen rellevants. Si $e=1$, triar $n$ vectors en dimensi\'o 1 ja \'es prou merda.)

Induction:
(We add a vector to the configuration) $\mathcal{V}(n + 1,e,m)$
Take a configuration $v = \{v_1, \ldots, v_n\}\in \mathcal{V}(n,e,m)$ for each vector $v_i$ in this configuration consider all the configurations that keep constant this $v_i$ and at all the other $v_j$ ($j \neq i$), we add the vectors of all the configurations of $\mathcal{V}(n,e,m)$ and the spare vector take as the $n+1$.

$\mathcal{V}(n,e + 1,m)$

(We incrementally consider larger boxes) $\mathcal{V}(n,e,m + 1)$
Assume we have generated all configurations in $\mathcal{V}(n,e,m)$, then the only new configurations are the ones with at least one vector of length $m+1$. So, for every $i \in [1,n]$, choose $i$ vectors in the boundary and $n-i$ as in $\mathcal{V}(n-i,e,m)$. It remains to be checked which vectors of the boundary can be avoided


\item
\end{enumerate}
\end{document}
